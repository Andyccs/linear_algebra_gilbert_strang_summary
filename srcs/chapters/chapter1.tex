\section{Vector and Matrices}

\subsection{Vectors and Linear Combinations}

\begin{itemize}
    \item Linear combination of vector $v$ and $w$ is $cv + dw$
    \item Linear combination of $v$ and $w$ fill the $xy$ plane in $\mathbb{R}^2$ if $v$
          and $w$ are linearly independent with dimension of 2 rows and 1 column
    \item Linear combination of $v$ and $w$ fill a plane in $\mathbb{R}^3$ if $v$ and $w$
          are linearly independent with dimension of 3 rows and 1 column
    \item Linear combination using matrix way:
          \[
              \begin{bmatrix}
                  v_1 & w_1 \\
                  v_2 & w_2
              \end{bmatrix} +
              \begin{bmatrix}
                  c \\
                  d
              \end{bmatrix} =
              \begin{bmatrix}
                  b_1 \\
                  b_2
              \end{bmatrix}
          \]
    \item Linear combination using column way:
          \[
              c \begin{bmatrix}
                  v_1 \\
                  v_2
              \end{bmatrix} +
              d \begin{bmatrix}
                  w_1 \\
                  w_2
              \end{bmatrix} =
              \begin{bmatrix}
                  b_1 \\
                  b_2
              \end{bmatrix}
          \]
    \item Linear combination using row way:
          \begin{align*}
              v_1 c + w_1 d & = b_1 \\
              v_2 c + w_2 d & = b_2
          \end{align*}

\end{itemize}

\subsection{Lengths and Angles from Dot Products}

\begin{itemize}
    \item The dot product of $\mathbf{v} = \begin{bmatrix} v_1 \\ v2 \end{bmatrix}$ and $\mathbf{w} =
              \begin{bmatrix} w_1 \\ w_2 \end{bmatrix}$ is defined as: $\mathbf{v} \cdot \mathbf{w} = v_1 w_1 + v_2 w_2$
    \item The length squared of $\mathbf{v}$ is defined as: $||\mathbf{v}||^2 =
              \mathbf{v} \cdot \mathbf{v} = v_1^2 + v_2^2$
    \item The length of v is defined as: $||v|| = \sqrt{\mathbf{v} \cdot \mathbf{v}} =
              \sqrt{v_1^2 + v_2^2}$
    \item The angle $\theta$ between two vectors $\mathbf{v}$ and $\mathbf{w}$ is defined
          as: $\cos \theta = \frac{\mathbf{v} \cdot \mathbf{w}}{||\mathbf{v}||
                  ||\mathbf{w}||}$
    \item Schwarz Inequality: $|\mathbf{v} \cdot \mathbf{w}| \leq ||\mathbf{v}||
              ||\mathbf{w}||$
    \item Triangle Inequality: $||\mathbf{v} + \mathbf{w}|| \leq ||\mathbf{v}|| +
              ||\mathbf{w}||$
\end{itemize}

\subsection{Matrices and Their Column Spaces}

\begin{itemize}
    \item Identity matrix
          \[
              \begin{bmatrix}
                  1 & 0 & 0 \\
                  0 & 1 & 0 \\
                  0 & 0 & 1
              \end{bmatrix}
          \]
    \item Diagonal matrix
          \[
              \begin{bmatrix}
                  2 & 0 & 0 \\
                  0 & 4 & 0 \\
                  0 & 0 & 5
              \end{bmatrix}
          \]
    \item Triangular matrix
          \[
              \begin{bmatrix}
                  2 & 1 & -3 \\
                  0 & 4 & 7  \\
                  0 & 0 & 5
              \end{bmatrix}
          \]
    \item Symmetric matrix
          \[
              \begin{bmatrix}
                  2  & 1 & -3 \\
                  1  & 4 & 7  \\
                  -3 & 7 & 5
              \end{bmatrix}
          \]
    \item Anti-symmetric matrix
          \[
              \begin{bmatrix}
                  0  & 1  & 3 \\
                  -1 & 0  & 7 \\
                  -3 & -7 & 0
              \end{bmatrix}
          \]
    \item Row picture of $A\mathbf{x}$ is dot products of $\mathbf{x}$ with rows of $A$
          \[
              \begin{bmatrix}
                  1 & 2 \\
                  3 & 4 \\
                  5 & 6
              \end{bmatrix}
              \begin{bmatrix}
                  7 \\
                  8
              \end{bmatrix} =
              \begin{bmatrix}
                  1 \cdot 7 + 2 \cdot 8 \\
                  3 \cdot 7 + 4 \cdot 8 \\
                  5 \cdot 7 + 6 \cdot 8
              \end{bmatrix}
          \]
    \item Column picture of $A\mathbf{x}$ is linear combinations of the columns of $A$
          \[
              \begin{bmatrix}
                  1 & 2 \\
                  3 & 4 \\
                  5 & 6
              \end{bmatrix}
              \begin{bmatrix}
                  7 \\
                  8
              \end{bmatrix} =
              7
              \begin{bmatrix}
                  1 \\
                  3 \\
                  5
              \end{bmatrix} +
              8
              \begin{bmatrix}
                  2 \\
                  4 \\
                  6
              \end{bmatrix}
          \]
    \item The column space $\mathbf{C}(A)$ contains all vector $A\mathbf{x}$, or all
          combinations of the columns of $A$
    \item The ``span'' of columns of $A$ is the column space
    \item When $A$ has 3 rows, its column space can be described as:
          \begin{itemize}
              \item A single point (the origin) if $A$ is zero matrix, rank $r=0$
              \item A line in $\mathbb{R}^3$ through the origin if $A$ has 1 independent column,
                    rank $r=1$
              \item A plane in $\mathbb{R}^3$ through the origin if $A$ has 2 independent columns,
                    rank $r=2$
              \item The whole space $\mathbb{R}^3$ if $A$ has 3 columns and they are linearly
                    independent, rank $r=3$
          \end{itemize}
\end{itemize}

\subsection{Matrix Multiplication $AB$ and $CR$}

Matrix multiplication:

\begin{itemize}
    \item Four ways to multiply $AB=C$
          \begin{itemize}
              \item Row way:
                    \[
                        A B =
                        \begin{bmatrix}
                            \text{row }1\text{ of }A \cdot \text{col }1\text{ of }B & \text{row }1\text{ of }A \cdot \text{col }2\text{ of }B \\
                            \text{row }2\text{ of }A \cdot \text{col }1\text{ of }B & \text{row }2\text{ of }A \cdot \text{col }2\text{ of }B
                        \end{bmatrix}
                    \]
              \item Column way: $A$ times column $j$ of $B$ produces column $j$ of $AB$
                    \[
                        AB =
                        \begin{bmatrix}
                            \\
                            Ab_1 & \cdots & Ab_n \\
                            \\
                        \end{bmatrix}
                    \]
              \item Row $i$ of $A$ times $B$
                    \[
                        AB =
                        \begin{bmatrix}
                            a_{i} B \\
                            \vdots  \\
                            a_{m} B
                        \end{bmatrix}
                    \]
              \item Column times row:
                    \[
                        AB =
                        a_{\text{col}=1} b_{\text{row}=1} + \cdots + a_n b_m
                    \]
          \end{itemize}
    \item $AB \neq BA$, but $A(BC) = (AB)C$ and $A(B + C) = AB + AC$
\end{itemize}

Matrix factorization $A = CR$:

\begin{itemize}
    \item Example of $A = CR$
          \[
              \begin{bmatrix}
                  1 & 2 & 3 \\
                  4 & 5 & 6 \\
                  7 & 8 & 9
              \end{bmatrix} =
              \begin{bmatrix}
                  1 & 2 \\
                  4 & 5 \\
                  7 & 8
              \end{bmatrix}
              \begin{bmatrix}
                  1 & 0 & -1 \\
                  0 & 1 & 2
              \end{bmatrix}
          \]
    \item If $A$ has $r$ independent columns in $C$, then $A = CR = (m \times r) (r
              \times n)$
    \item $C$ contains the first r independent columsn of $A$
    \item $R$ tells how to produce all columns of A from the columns of $C$
    \item Matrix $R$ is also known as reduced row echelon form (RREF) of $A$
    \item Another way to look at $A = CR$:
          \begin{align*}
              R & = \begin{bmatrix}
                        I & F
                    \end{bmatrix}                                                  \\
              A & = C \begin{bmatrix}
                          I & F
                      \end{bmatrix}                                                \\
                & = \begin{bmatrix}
                        C & CF
                    \end{bmatrix}                                                  \\
                & = \begin{bmatrix}
                        \text{independent columns of A} & \text{dependent columns of A}
                    \end{bmatrix}
          \end{align*}
\end{itemize}
