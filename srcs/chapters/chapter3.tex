\section{The Four Fundamental Subspace}

\subsection{Vector Spaces and Subspaces}

\begin{itemize}
    \item The vector space $\mathbb{R}^n$ contains all column vectors $v$ of length $n$
          and the component $v_1$ to $v_n$ are real numbers.
    \item The linear combination of vector $\mathbf{v}$ and $\mathbf{w}$ in
          $\mathbb{R}^n$, $c\mathbf{v} + d\mathbf{w}$ must stay in the vector space
          $\mathbb{R}^n$.
    \item A vector space does not have to only contain ``vector''. It is possible to have
          a vector space of matrices and vector spaces of functions, as long as these
          vector spaces obey the \textbf{eight axioms of vector spaces}.
    \item A subspace is a vector space inside a vector space. A subspace containing
          $\mathbf{v}$ and $\mathbf{w}$ must contain all linear combinations $c
              \mathbf{v} + d \mathbf{w}$.
    \item The column space consists of all linear combinations of the columns. Those
          combinations are all possible vectors $A\mathbf{x}$. They fill the column space
          $\mathbf{C}(A)$. The equations $A\mathbf{x} = \mathbf{b}$ are solvable if and
          only if $\mathbf{b}$ is in the column space of $A$.
    \item The row space of $A$ is the column space $\mathbf{C}(A^T)$
\end{itemize}

\noindent Eight axioms of vector spaces:
\begin{itemize}
    \item $\mathbf{x} + \mathbf{y} = \mathbf{y} + \mathbf{x}$
    \item $\mathbf{x} + (\mathbf{y} + \mathbf{z}) = (\mathbf{x} + \mathbf{y}) + \mathbf{z}$
    \item There is a unique zero vector such that $\mathbf{x} + 0 = \mathbf{x}$ for all
          $\mathbf{x}$
    \item For each $\mathbf{x}$ there is a unique vector $-\mathbf{x}$ such that
          $\mathbf{x} + (-\mathbf{x}) = 0$
    \item 1 times $\mathbf{x}$ equals to $\mathbf{x}$
    \item $(c_1 c_2) \mathbf{x} = c_1 (c_2 \mathbf{x})$
    \item $c (\mathbf{x} + \mathbf{y}) = c \mathbf{x} + c \mathbf{y}$
    \item $(c_1 + c_2)\mathbf{x} = c_1 \mathbf{x} + c_2 \mathbf{x}$
\end{itemize}

\subsection{The Nullspace of $A$: Solving $A\mathbf{x} = 0$}

Example of Elimination from $A$ to rref($A$): Reduced Row Echelon Form $R_0$

\[
    A =
    \begin{bmatrix}
        1 & 7  & 3 & 35 \\
        2 & 14 & 6 & 70 \\
        2 & 14 & 9 & 97
    \end{bmatrix}
    \rightarrow
    \begin{bmatrix}
        1 & 7 & 3 & 35 \\
        0 & 0 & 0 & 0  \\
        0 & 0 & 3 & 27
    \end{bmatrix}
    \rightarrow
    \begin{bmatrix}
        1 & 7 & 0 & 8  \\
        0 & 0 & 0 & 0  \\
        0 & 0 & 3 & 27
    \end{bmatrix}
    \rightarrow
    \begin{bmatrix}
        1 & 7 & 0 & 8 \\
        0 & 0 & 1 & 9 \\
        0 & 0 & 0 & 0
    \end{bmatrix} = R_0
\]

\noindent Removing the zero rows produce $R$
\[
    R_0 =
    \begin{bmatrix}
        1 & 7 & 0 & 8 \\
        0 & 0 & 1 & 9 \\
        0 & 0 & 0 & 0
    \end{bmatrix}
    \rightarrow
    \begin{bmatrix}
        1 & 7 & 0 & 8 \\
        0 & 0 & 1 & 9
    \end{bmatrix}
    = R
\]

\noindent Three allowed row operations in elimination from $A$ to $R_0$:
\begin{itemize}
    \item Subtract a multiple of one row from another row (below or above)
    \item Divide a row by its first nonzero entry
    \item Exchange rows
\end{itemize}

\noindent By permuting the columns of $R$, we get $R = [\ I\ F\ ]$
\[
    R =
    \begin{bmatrix}
        1 & 7 & 0 & 8 \\
        0 & 0 & 1 & 9
    \end{bmatrix}
    =
    \begin{bmatrix}
        1 & 0 & 7 & 8 \\
        0 & 1 & 0 & 9
    \end{bmatrix}
    \begin{bmatrix}
        1 & 0 & 0 & 0 \\
        0 & 0 & 1 & 0 \\
        0 & 1 & 0 & 0 \\
        0 & 0 & 0 & 1 \\
    \end{bmatrix}
    =
    \begin{bmatrix}
        I & F
    \end{bmatrix} P_C
\]

\noindent Another way to look at the matrix $A = CR$: is
\begin{align*}
    R & = \begin{bmatrix}
              I & F
          \end{bmatrix}                                                  \\
    A & = C \begin{bmatrix}
                I & F
            \end{bmatrix}                                                \\
      & = \begin{bmatrix}
              C & CF
          \end{bmatrix}                                                  \\
      & = \begin{bmatrix}
              \text{independent columns of A} & \text{dependent columns of A}
          \end{bmatrix}
\end{align*}

\noindent In the $R_0$ example above, columns 2 and columns 4 give the ``free variable''.
The free variables allow use to compute the special solutions for $R_0
    \mathbf{x} = 0$ or $A \mathbf{x} = 0$
\begin{itemize}
    \item Set $x_2 = 1$ and $x_4 = 0$, we get $s_1 = (-7, 1, 0, 0)$
    \item Set $x_2 = 0$ and $x_4 = 1$, we get $s_2 = (-8, 0, -9, 1)$
    \item All $\mathbf{x}$ that have $A \mathbf{x} = 0$ are $\mathbf{x}_n = c_1 s_1 + c_2
              s_2$, which is also the nullspace of $A$ or $\mathbf{N}(A)$ in $\mathbb{R}^n$
    \item Elimination from $A$ to $R_0$ to $R$ does not change the nullspace
          $\mathbf{N}(A) = \mathbf{N}(R_0) = \mathbf{N}(R)$
\end{itemize}

\noindent Block elimination in three steps:
\[
    P_R A P_C =
    \begin{bmatrix}
        W & H \\
        J & K
    \end{bmatrix}
    \rightarrow
    \begin{bmatrix}
        I & W^{-1}H \\
        J & K
    \end{bmatrix}
    \rightarrow
    \begin{bmatrix}
        I & W^{-1}H \\
        0 & 0
    \end{bmatrix}
    = R_0
\]
\[
    P_R A P_C =
    \begin{bmatrix}
        W \\
        J
    \end{bmatrix}
    W^{-1}
    \begin{bmatrix}
        W & H
    \end{bmatrix}
\]
\begin{itemize}
    \item $W$ is $r$ by $r$ invertible matrix
    \item The block satisfy $JW^{-1}H = K$
\end{itemize}

\subsection{The Complete Solution to $A\mathbf{x} = \mathbf{b}$}

\subsection{Independence, Basis, and Dimension}

\subsection{Dimensions of the Four Subspaces}