\section{Solving Linear Equations $Ax = b$}

\subsection{Elimination and Back Substitution}

Elimination change $Ax = b$ to $Ux = c$, example:

\begin{alignat*}{3}
     & \quad                    &  & A        = \begin{bmatrix}
                                                    2 & 3  & 4  \\
                                                    4 & 11 & 14 \\
                                                    2 & 8  & 17
                                                \end{bmatrix}       \quad                                &  & b                  = \begin{bmatrix}
                                                                                                                                       19 \\
                                                                                                                                       55 \\
                                                                                                                                       50
                                                                                                                                   \end{bmatrix} \\
     & E_{21} = \begin{bmatrix}
                    1  & 0 & 0 \\
                    -2 & 1 & 0 \\
                    0  & 0 & 1
                \end{bmatrix} \quad &  & E_{21}A                     = \begin{bmatrix}
                                                                           2 & 3 & 4  \\
                                                                           0 & 5 & 6  \\
                                                                           2 & 8 & 17
                                                                       \end{bmatrix} \quad             &  & E_{21}b  = \begin{bmatrix}
                                                                                                                           19 \\
                                                                                                                           17 \\
                                                                                                                           50
                                                                                                                       \end{bmatrix}             \\
     & E_{31} = \begin{bmatrix}
                    1  & 0 & 0 \\
                    0  & 1 & 0 \\
                    -1 & 0 & 1
                \end{bmatrix} \quad &  & E_{31}E_{21}A                     = \begin{bmatrix}
                                                                                 2 & 3 & 4  \\
                                                                                 0 & 5 & 6  \\
                                                                                 0 & 5 & 13
                                                                             \end{bmatrix} \quad       &  & E_{31}E_{21}b  = \begin{bmatrix}
                                                                                                                                 19 \\
                                                                                                                                 17 \\
                                                                                                                                 31
                                                                                                                             \end{bmatrix}       \\
     & E_{32} = \begin{bmatrix}
                    1 & 0  & 0 \\
                    0 & 1  & 0 \\
                    0 & -1 & 1
                \end{bmatrix} \quad &  & E_{32}E_{31}E_{21}A                     = \begin{bmatrix}
                                                                                       2 & 3 & 4 \\
                                                                                       0 & 5 & 6 \\
                                                                                       0 & 0 & 7
                                                                                   \end{bmatrix} \quad &  & E_{32}E_{31}E_{21}b  = \begin{bmatrix}
                                                                                                                                       19 \\
                                                                                                                                       17 \\
                                                                                                                                       174
                                                                                                                                   \end{bmatrix} \\
\end{alignat*}
\begin{align*}
    l_{21} = 2, l_{31} = 1, l_{32} = 1 & , \text{will be useful to compute }E^{-1} \\
    E_{32}E_{31}E_{21}A = U            & , E \text{ transforms } A \text{ to } U   \\
    E_{32}E_{31}E_{21}b = c            & , E \text{ transforms } b \text{ to } c   \\
    Ax = b \rightarrow Ux = c                                                      \\
    x = (4, 1, 2)                      & , \text{by back substitution}             \\
\end{align*}

\subsection{Elimination Matrices and Inverse Matrices}

Elimination Matrices:
\begin{itemize}
    \item The elimination matrix is $E = E_{32} E_{31} E_{21}$. $EA = U$ and $U$ is an
          upper triangular matrix
    \item Using example from section 2.1, the inverse of elimination matrix $E^{-1}$ is
          \[
              E^{-1} =
              \begin{bmatrix}
                  1      &        &   \\
                  l_{21} & 1      &   \\
                  l_{31} & l_{32} & 1 \\
              \end{bmatrix}
          \]
    \item Since $EA = U$, so $A = E^{-1} U$ and $A = LU$
\end{itemize}

\noindent Inverse Matrices:
\begin{itemize}
    \item The matrix $A$ is invertible if there exists a matrix $A^{-1}$ that inverts
          $A$. $A^{-1}A = I$ and $AA^{-1} = I$
    \item If $A$ is invertible, then $Ax = b \rightarrow x = A^{-1}b$
    \item 2 by 2 matrix is invertible if $ad-bc \neq 0$
          \[
              \begin{bmatrix}
                  a & b \\
                  c & d
              \end{bmatrix}^{-1} =
              \frac{1}{ad-bc}
              \begin{bmatrix}
                  d  & -b \\
                  -c & a
              \end{bmatrix}
          \]
    \item The inverse of product $AB$ is $(AB)^{-1} = B^{-1} A^{-1}$
\end{itemize}

\subsection{Matrix Computation $A = LU$}

Skipped

\subsection{Permutations and Transposes}

Permutations:
\begin{itemize}
    \item A row permutation matrix $P$ has the same rows as $I$ in any order
    \item $P^T = P^{-1}$
    \item Some time you need a row permutation P to order A so that no zeroes in the
          pivot positions during elimination. So $PA = LU$
    \item A column permutation matrix $Q$ has the same columns as $I$ in any order.
          $PAQ$, column permutation on the left, row permutation on the right
\end{itemize}

\noindent Transpose:
\begin{itemize}
    \item $(A + B)^T = A^T = B^T$
    \item $(AB)^T = B^T A^T$
    \item $(A^{-1})^T = (A^T)^{-1}$
\end{itemize}

\noindent Symmetric Products:
\begin{itemize}
    \item For any matrix $A$, $A^T A$ and $A A^T$ must be symmetric
    \item Symmtric matrices can be factorized into $LDL^T$
          \begin{align*}
              S = LDL^T \\
              \begin{bmatrix}
                  1 & 2 \\
                  2 & 7
              \end{bmatrix} =
              \begin{bmatrix}
                  1 & 0 \\
                  2 & 1
              \end{bmatrix}
              \begin{bmatrix}
                  1 & 0 \\
                  0 & 3
              \end{bmatrix}
              \begin{bmatrix}
                  1 & 2 \\
                  0 & 1
              \end{bmatrix}
          \end{align*}
    \item For saddle-point matrix S
          \begin{align*}
              S               & = LDL^T \\
              \begin{bmatrix}
                  I   & A \\
                  A^T & 0
              \end{bmatrix} & =
              \begin{bmatrix}
                  I   & 0 \\
                  A^T & I
              \end{bmatrix}
              \begin{bmatrix}
                  I & 0      \\
                  0 & -A^T A
              \end{bmatrix}
              \begin{bmatrix}
                  I & A \\
                  0 & I
              \end{bmatrix}
          \end{align*}
\end{itemize}

\subsection{Derivatives and Finite Difference Matrices}

Skipped
